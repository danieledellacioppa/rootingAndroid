\def\tetraexample#1#2#3#4#5#6#7#8#9{
\begin{figure}[h]
\centering
\begin{tikzpicture}[node distance=5mm,
terminal/.style={
% The shape:
rectangle,
minimum size=0.1\linewidth,
% The rest
very thick,draw=white,
top color=white,bottom color=white,
font=\rmfamily},every annotation/.style={
minimum size=0.05\linewidth,
fill=BrickRed!45}]
\node (russell) [terminal] at (-8,0)
{
\lstinputlisting#1
};

\node (highlight) [rectangle,
    draw = black,
    thick,
    text = black,
%    path fading=north,
    rounded corners=1mm,
    fill = black,
    opacity=.2,
    minimum width = 56ex, 
    minimum height = #7,
    yshift=#8] at (russell.north)
{

};

\node (centro) [rectangle,
    draw = BrickRed,
    thick,
    text = black,
%    path fading=north,
    rounded corners=1mm,
    fill = BrickRed,
    opacity=.2,
    #9] at (highlight.center)
{

};

\node[inner sep=0pt,xshift=7cm] (whitehead) at (highlight.east)
    {#2};
\draw[->,thick] (highlight.east) -- (whitehead.west)
    #3

%\node (highlight) [annotation,scale=1,yshift=-9ex,xshift=8ex,path fading=north] at (russell.north)
%{
%
%};

\node (explaining) [annotation,scale=2,yshift=-1cm,xshift=-5ex,path fading=south,fill=BrickRed!30] at (whitehead.south)
{

{\fontsize{4.5pt}{2pt}\selectfont #5} 

};

\draw [->,thick,BrickRed] (centro.south) .. controls +(down:5mm) and +(left:5mm) .. (explaining.west)
            #4

\end{tikzpicture}
#6
\end{figure}
}

